%%%%%%%%%%%%%%%%%%%%%%%%%%%%%%%%%%%%%%%%%%%%%%%%%%%%%%%%
% Este é um documento que servirá de modelo para
% os relatórios feitos na disciplina Circuitos Digitais
% 2016-2
%%%%%%%%%%%%%%%%%%%%%%%%%%%%%%%%%%%%%%%%%%%%%%%%%%%%%%%%%

\documentclass[12pt]{article}

\usepackage{sbc-template}
\usepackage[brazil,american]{babel}
\usepackage[utf8]{inputenc}

\usepackage{graphicx}
\usepackage{url}
\usepackage{float}
\usepackage{listings}
\usepackage{color}
\usepackage{todonotes}
\usepackage{algorithmic}
\usepackage{algorithm}
\usepackage{hyperref}
     
\sloppy


\title{Trabalho 1\\ 
Processos Iterativos\\ 
\&\\
Solução de equações não-lineares}

\author{Dayanne Fernandes da Cunha, 13/0107191\\
       Yurick Hauschild , 12/0024136
}


\address{Dep. Matemática -- Universidade de Brasília (UnB)\\
  Cálculo Numérico - Turma A
  \email{dayannefernandesc@gmail.com, yurick.hauschild@gmail.com}
}

\begin{document} 
\maketitle

\selectlanguage{american}
 \begin{abstract}
	This report corresponds to the ...
 \end{abstract}
\selectlanguage{brazil}     
    
 \begin{resumo} 
 	Este relatório corresponde aos informativos das resoluções do Trabalho 1 de Cálculo Numérico da Turma A do semestre 2016/2.
 \end{resumo}

\section*{Parte I: Processos iterativos}
\label{sec:parte1}

Esta primeira questão será sobre as bifurcações do mapa logístico. Considere o processo iterativo da Equação~\ref{eq:parte1}, chamado de \textit{mapa logístico}. Este processo iterativo, apesar de aparentar ser bastante simples, tem uma dinâmica muito rica, que será analisada em detalhes ao longo desta parte 1 do trabalho.

\begin{equation}
x_{n+!} = \lambda x_{n}(1 - x_{n})
\label{eq:parte1}
\end{equation}

\subsection*{Questão 1}
\label{sec:p1q1}
Determine analiticamente pontos fixos $x^{*}$ do mapa logístico, Equação~\ref{eq:parte1} e determine as condições para que sejam assintoticamente estáveis. Veja que o parâmetro crucial deste problema é $\lambda$ .

\section*{Parte II: Solução de equações não-lineares}
\label{sec:parte2}

\end{document}
